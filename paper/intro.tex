\section{Introduction}

The State-Machine Replication technique (abbreviated SMR) allows building
reliable services on top of unreliable hardware. The basic idea is to
replicate a service over several servers, each executing the same sequence of
deterministic commands. In such a setup, any replica can be used to determine
the result of a command. Therefore, if a replica crashes, clients can be
transparently redirected to another one and will have the illusion of accessing
a centralized service at all times.

Making SMR algorithms efficient has been a long-standing challenge
and a multitude of algorithms have been proposed. Examples include
Paxos \cite{lamport2001paxos}, Fast Paxos \cite{Lamport06FastPaxos},
Disk Paxos \cite{GafniLamport03DiskPaxos}, Chain Replication
\cite{RenesseSchneider04ChainReplicationSupportingHighThroughputAvailability},
Ring Paxos \cite{MarandiETAL10RingPaxosHighthroughputAtomicBroadcastProtocol},
Egalitarian Paxos
\cite{MoraruAndersenKaminsky13ThereIsMoreConsensusEgalitarianParliaments},
Multi-Coordinated Paxos \cite{CamargosSchmidtPedone07MulticoordinatedPaxos},
Cheap Paxos \cite{LamportMassa04CheapPaxos}, Vertical Paxos
\cite{LamportMalkhiZhou09VerticalPaxosPrimarybackupReplication}, Paxos-MIC
\cite{HurfinMoiseNarzul11AdaptiveFastPaxosMakingQuickEverlasting}, Mencius
\cite{MaoJunqueiraMarzullo08MenciusBuildingEfficientReplicatedStateMachine},
Fast Mencius \cite{WeiETAL13FastMenciusMenciusLowCommitLatency},
the Chandra-Toueg algorithm
\cite{ChandraToueg96UnreliableFailureDetectorsReliableDistributedSystems}, the
Ben-Or algorithm \cite{BenOr83AnotherAdvantageFreeChoiceCompletelyAsynchronous},
and Raft \cite{OngaroOusterhout14SearchUnderstandableConsensusAlgorithm}.

With the wide use of cloud computing, improvements in the performance and
efficiency of SMR algorithms have the potential to impact most online services:
advanced SMR algorithms can be implemented by cloud computing service providers
and benefit all their customer's applications.

However, devising new SMR algorithms, optimized with
particular situations or metrics in mind, is hard: today,
more than two decades after the first SMR algorithms
\cite{Lamport98ParttimeParliament,BirmanJoseph87ReliableCommunicationPresenceFailures,OkiLiskov88ViewstampedReplicationGeneralPrimaryCopy,DworkLynchStockmeyer84ConsensusPresencePartialSynchronyPreliminaryVersion} were proposed, devising new
SMR algorithms is still complex enough to justify a publication in a top conference
\cite{MoraruAndersenKaminsky13ThereIsMoreConsensusEgalitarianParliaments,OngaroOusterhout14SearchUnderstandableConsensusAlgorithm}.

The problem is that it is difficult to reuse existing algorithms and ideas because
SMR algorithms are not modular. When devising a new algorithm, all the careful
analysis, tests, and proofs have to be done again from scratch.

We propose a framework to make existing Generalized Consensus algorithms modular
and to enable adding new optimizations incrementally, without modification to
the existing. Generalized consensus, abbreviated GC, is a precise formulation of
State-Machine Replication that subsumes consensus. An algorithm in our framework
is composed of independent modules that can run, abort their execution, and pass
the baton to any other module on the fly.

Crucially, when adding a new module to an existing algorithm, all the
performance analysis, tests, and proofs of the existing modules remain valid and
only the new GC module needs to be analyzed: Our composition theorem guarantees
that the composition of any number of correct modules is a correct GC algorithm,
thus one can combine different GC algorithms, optimized for different situations
and metrics, without any changes and with guaranteed correctness. Our framework
therefore enables incremental development of GC algorithms, where each new
development can build on the previous ones instead of starting again from
scratch.

Moreover, our framework enables a clear separation of concerns which simplifies
the design of new individual modules. A composite GC algorithm built with our
framework works by running the GC module deemed the most appropriate at any
given time, dynamically changing GC module when needed. Therefore a composable
GC algorithm is not required to make progress in all situations: if it does
not make progress, the system can switch to another GC module. Thus algorithms
designers who want to create a new GC algorithm, optimized for particular
conditions, can focus only on the particular conditions of interest, knowing
that the other existing modules will handle the other cases.

The key novelty of our framework lies in its practical applicability:
\begin{itemize}
  \item Our framework is based on Generalized Consensus, a generalization of the consensus problem 
    that allows optimizing the execution of commuting commands.
  \item Many SMR algorithms are variations on a basic structure consisting of one of two kinds of 
    quorum systems \cite{GuerraouiVukolic10RefinedQuorumSystems}. 
    The first kind, that we call resilient quorums, are used in leader-based algorithms, while the second
    kind of quorums, that we call fast quorums, are used in leader-less algorithms and can order commuting commands without synchronization.
    We provide two specifications, at an intermediate level of abstraction, that capture the essence of composing algorithms using these two kinds of quorums.
    Our two specifications allow to easily check that new modules based on resilient or fast quorums are correct, covering a large spectrum of practical algorithms. 
\end{itemize}

\begin{comment}
Allowing the execution of commuting commands to proceed without
synchronization is an important practical optimization, as commands may
often commute in practice, for example when operating on disjoint parts
of a data structure. To enable the synchronization-free execution of
commuting commands, we have based our framework on Generalized Consensus
\cite{Lamport05GeneralizeConsensus}. Traditional SMR implementations use a
separate instance of a consensus algorithm to agree on the command to execute
at each index of the command sequence on which the replicas must agree. Using
separate consensus instances makes it hard to optimize the execution of
commuting command because commutativity is not a property of a single command.
Instead of restricting agreement to a single value, the Generalized Consensus
problem requires that set of replica servers agree on the state of a growing
data structure called a command-structure set. A command-structure set is
a data structure with an append operation such that the order in which two
commuting commands are appended does not change the resulting state. Thanks to
this property, two replicas executing commuting commands in a different order
will not end up in a different state and can continue executing without further
synchronization. In message-passing systems, this property can be leveraged to
reduce the latency of processing one command to two message delays instead of
three. 
%As we will see, fast execution of commutative requests also enables the processing of read-only commands
%in two message delays, although read-only commands do not commute with write
%commands.

In practice, a SMR algorithm must support fast reconfiguration to replace
faulty components, without which the long-term stability of a system is
inevitably compromised. Our framework abstracts over how the reconfiguration
decisions are made, providing an interface allowing to plug in an external
reconfiguration master implementing the reconfiguration policy of the
system. The master decides which GC module to run next and which replica
servers will participate, removing failed replicas and adding new ones. The
reconfiguration master enables tolerating $k$ failures with $k+1$ replicas
\cite{LamportMalkhiZhou09VerticalPaxosPrimarybackupReplication}, an important
improvement over the traditional $2k+1$ needed replicas.
\end{comment}

To demonstrate the practical applicability of our framework we provide
composable versions of the generalized variations of popular consensus
algorithms: Classic Paxos, Fast Paxos, Chain Replication, and multi-coordinated
Paxos. Thanks to our specifications of algorithms based on quorum-systems, 
we have checked the correctness of our version of Fast Paxos and Chain Replication with the TLC
model-checker and we provide all the specifications needed to replicate these
experiments. Our framework is also fully formalized and its composition theorem
proved in Isabelle/HOL, a foundational interactive theorem prover, thereby
certifying the correctness of the composition theorem at the deepest level.
The TLA+ and Isabelle/HOL development can be found at \url{http://losa.fr/cgc}.

At the heart of our framework lies an abstraction, called Composable Generalized
Consensus (CGC), of the familiar notion of round (also called epoch, ballot, or
view in the literature) used in most SMR algorithms. As an example, let us
consider Fast Paxos, one of the first SMR algorithm which can change strategy
dynamically. Fast Paxos is a round-based SMR algorithm that has two types of
rounds: fast rounds and normal rounds. Fast rounds process requests faster
than normal rounds in contention-free executions but fall back to a slower
scheme under contention. Normal rounds are the same as in the classic Paxos
and are slower than fast rounds in contention-free executions. Any process can
initiate a round change at any point during execution: this allows switching
between fast and normal rounds as well as changing the leader from one normal
round to the next, similarly to classic Paxos. Each kind of round maintains
different invariants relating the state of each process and the values that
might have been chosen in the previous rounds, thus a given round must know
which past rounds are fast or normal rounds in order to take over the execution
safely. When adding more kinds of rounds such an approach will quickly become
unmanageable, as each type of rounds must have one case for each possible other
kind of round, making $O(n^2)$ cases to consider in total.

Our notion of Composable Generalized Consensus defines a common interface and a
common correctness property for a round, abstracting over the particular round
implementations and making all types of rounds compatible with each other.

Moreover, our interface for rounds enables a crucial optimization seen in
practical systems: introducing an external reconfiguration master which manages
the addition of new replicas and the removal of failed ones. Interestingly, as
explained in \cite{LamportMalkhiZhou09VerticalPaxosPrimarybackupReplication},
in a real system it is often possible to implement a reliable reconfiguration
master which allows to tolerate $f$ faults with $f+1$ replicas instead of the
traditional $2f+1$ replicas. The reconfiguration master can also make the choice
of the new type of round to execute, providing the system's reconfiguration
policy.
