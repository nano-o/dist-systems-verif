\section{Related Work}
\label{sec:related}

% Abstract framework
Our work is closely related to the Abstract
framework~\cite{GuerraouiETAL10Next700BftProtocols}
and the Speculative Linearizability
framework~\cite{GuerraouiKuncakLosa12SpeculativeLinearizability}. The
Abstract framework pioneered the idea of decomposing SMR algorithms in
independent modules and experimentally showed that the resulting composite
algorithms outperformed state-of-the-art monolithic algorithms. Compared
to our work, the Abstract framework is not precisely formalized and lacks
tools supporting the development of correct algorithms in the framework,
which are direly needed given the notorious complexity of SMR algorithms.
Moreover, the Abstract framework does not allow optimizing for the execution
of commuting commands. However, the Abstract framework supports byzantine
fault-tolerant~\cite{LamportShostakPease82ByzantineGeneralsProblem} SMR
algorithms, which we do not. 

% Speclin
The Speculative Linearizability framework was a
first attempts by the authors to formalize the Abstract framework, provide
formal proof support in Isabelle/HOL~\cite{NipkowPaulsonWenzel02IsabelleHOL},
and extend the Abstract framework to support algorithms that optimize
the execution of commuting commands by requiring Speculatively
Linearizable modules to refine Abstract modules (under a forward/backward
simulation~\cite{LynchVaandrager95ForwardBackwardSimulationsIUntimedSystems})
modulo an equivalence relation on histories of commands that reflects the
commutativity relation among commands. However, the simulation technique led to
complex correctness statements that were hard to scale beyond the examples of
the paper, and no realistic protocol was built in the framework. The present
work addresses the shortcoming of the Speculatively Linearizable framework by
using the notion of command-structure set introduced for the definition of
Generalized Consensus~\cite{Lamport05GeneralizeConsensus}. Moreover, we show how
several well known and practical algorithms can be cast and optimized in our
framework, providing a set of TLA+ specifications that ease the design and model
checking of algorithms based on quorum systems.

% Generalized Paxos
Generalized Consensus is an agreement problem relaxing the ordering constraints
imposed on commuting commands and Generalized Paxos~\cite{Lamport05GeneralizeConsensus} is an algorithm, similar
in structure to Fast Paxos, which takes advantage of the relaxed ordering
constraint to process commuting commands in two message delays, when three are
needed in the Classic Paxos algorithm.

Generalized Paxos is also structured in a sequence of rounds, 
where each round maintains the agreement property of Generalized Consensus.
In the abstract presentation of Generalized Paxos, Lamport gives a generic and abstract algorithm (not a distributed algorithm, accesses directly the state of all acceptors and rounds) that a round must use to determine the values that are safe to be voted for in the round.
If all round refine this abstract specification, then the resulting algorithm implements Generalized Consensus. Therefore any set of rounds refining Lamport's abstract specification can be combined to form a correct generalized consensus algorithm.  
However, no explicit interface for combining rounds is given and specifying a type of round in isolation from the others would require assume-guarantee reasoning. In contrast, our formulation uses encapsulared rounds and does not require assume-guarantee reasoning.

% Vertical Paxos
Vertical Paxos algorithms are SMR algorithms whose set of replica servers can
be reconfigured without stopping to process new commands. Crashed servers can
therefore be replaced by new servers, ensuring the long-term availability of a
replicated service. Vertical Paxos algorithms use an external reconfiguration
master, itself implemented as a replicated state machine, which determines
the servers that should be removed from or added to the configuration. The
role of the reconfiguration master is similar to the role of the scheduling
policy in our framework. In contrast to our work, Vertical Paxos does not define
an interface and correctness condition for encapsulating the reconfigurable
instances of SMR algorithms and therefore does not allow the reuse of existing
algorithms and the incremental development of SMR optimizations.


